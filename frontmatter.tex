% Title Page
\begin{titlepage}
  \thispagestyle{empty}
  \centering
  \vspace*{\fill}
  {\LARGE \textbf{\expandafter{\thesistitleI}}}\\[2mm]
  %{\LARGE \textbf{\expandafter{\thesistitleII}}}
  \vfill
  {\scshape by}\\
  {\large \expandafter{\myname}}\\
  \vfill
  {\scshape a thesis submitted to}\\
  \textbf{The Faculty of Graduate and Postdoctoral Affairs}\\
  {\scshape in partial fulfillment of the requirements for the degree of}\\
  \textbf{Doctor of Philosophy}\\
  {\scshape in}\\
  \textbf{Biology}
  \vfill
  {\large Carleton University}\\
  {\large Ottawa, Ontario, Canada}
  \vfill
  {\large \textcopyright{} 2024}\\
  {\large \expandafter{\myname}}
\end{titlepage}

\clearpage

% Info Page
\setcounter{page}{2}

\chapter*{Abstract}

\addcontentsline{toc}{chapter}{Abstract}

\par In biological surveys, detectability refers to the observer's ability to detect an organism, and is typically expressed as a probability of detection. 
Often, detectability is ignored in conservation problems that draw from biological surveys, because the survey lacks ancillary information to estimate detectability, or detectability is simply unknown for a particular species.
This can be problematic in instances where conservation managers may want to trade off monitoring versus action for a particular species, because it is impossible to know if zeros in a biological survey are true zeros because the species was not present, or a false zero because the species was there but not detected.
In estimates of population sizes, detectability is used to scale observed abundance of an organism to true abundance or density.
\par North American birds have been studied and surveyed extensively; as such, there is a plethora of point count data that are available across several programs.
Because of this, we are in an excellent position to estimate detectability of many landbird species using data-driven approaches, such that these detectability estimates may be used for future conservation purposes.
In this thesis, I focus on this data-driven estimation of detection probabilities for over 300 species of North American landbirds, by making use of the millions of data points available throughout North America.
In Chapter 2, I lay out an overview of how the detection process works in birds, and how to model detectability.
In Chapter 3, I then provide a pipeline for data-driven detectability estimates, by using methodology developed by the Boreal Avian Modelling project.
This allowed for detection probabilities to be estimated for 338 species of North American landbirds.
I expand upon these species in Chapter 4 by demonstrating how the use of Bayesian hierarchical modelling can help ``fill in the gaps" for detection probabilities for data-sparse species.
In Chapter 5, I apply these detection probabilities to the North American Breeding Bird Survey, a long-running bird survey that has not previously been able to account for detectability in their trend analysis.
I show that by accounting for changes in detectability due to changes in the landscape at the local scale, trends can change dramatically from near-flat to highly decreasing.
Finally, in Chapter 6, I examine the implications for conservation that the availability of detectability estimates can have, which include implications for data integration.

\chapter*{Acknowledgements}

\addcontentsline{toc}{chapter}{Acknowledgements}

\par I've got a lot of thanks for you people, and now, you're gonna hear about it! 
To say that this entire thesis is a product of all the wonderful people supporting me, both now and in the past, would be a huge understatement.

\par I want to first start by thanking my undergraduate mentors, Dr Daniel Gillis and Dr Shoshanah Jacobs.
Dan, you took such a huge risk on me when I waltzed into your office as an eager first-year computer science student in my second semester, and hired me on as an undergraduate research assistant.
Not only did you let me help out with your students' projects, but you gave me the opportunity to build my own projects based on my interest in birds, and you let me experience the life of an academic before even getting into grad school.
Sho, thank you for all of your biological wisdom in helping getting that piping plover project away from a super nerdy computer science exercise to an actual biologically-relevant project. 
Both of you have had a profound effect on my career as a researcher, and I will be forever grateful for all the opportunities you allowed for me.

\par An absolutely huge thank you to my PhD supervisors Dr Joseph Bennett and Dr Adam Smith.
Adam, I appreciate you giving me the opportunity to do an co-op position with you during my undergraduate, which ended up turning into a super cool opportunity to work with Breeding Bird Survey data and all the groups that are interested in bird monitoring.
Thank you so much for all your statistical wizardry over the years; I always appreciated our meetings that mostly involved staring at Stan code until the "ah-ha" moments.
Joe, thank you for also taking the chance on co-supervising me, fresh out of an undergraduate degree in math and stats with the desire to do a PhD in biology.
You have allowed me the opportunity to truly develop a transdisciplinary skill set as a statistical ecologist in the world of conservation science.
I feel like I have learned more about the state of conservation from you than I ever will in my lifetime.
To both of you, thank you for giving me the freedom to work at my own pace, and for having the wherewithal to reign me in when I go off the tracks.

\par This thesis has been supported by a plethora of people from quite literally around the world. 
Thank you to my thesis committee members Dr Lenore Fahrig and Dr Heather Kharouba (University of Ottawa).
I always have appreciated having your perspectives during my committee meetings, and you have encouraged me always to think beyond the model outputs in my work.
Thank you to Dr Elly Knight, Dr Erin Bayne, Dr Diana Stralberg, and the many other wonderful scientists at the Boreal Avian Modelling project for allowing me to develop a project to bring to Edmonton for a research visit.
I had such a great time visiting you all, and it was a great experience to work with the people that came up with many of the methods I used here.
Thank you to Dr David Miller (Biomathematics and Statistics Scotland) and Dr Alison Johnston (University of St. Andrews) for also allowing me to develop a project to bring to St Andrews for a research visit.
Never in my wildest dreams did I think I would have a chance to visit such a cool area in the world as St Andrews, let alone work with some brilliant people at CREEM.
I really do miss the cake culture at CREEM, and that's something I will plan to bring to my future workplace.
Thank you to the entirety of the Partners in Flight Population Estimates subcommittee.
It has been an absolute privilege to work with all of you over the last 4 and a half years, and I have learned so much.

\par Friends and family



\chapter*{Thesis Format and Co-authorship}

\addcontentsline{toc}{chapter}{Thesis Format and Co-authorship}

\par This thesis is an original work by Brandon Edwards.
Much of the research in this thesis, however, was conducted as part of various collaborations.
As such, all chapters are my own original work supported by co-authors as listed below.
I conducted all conceptualization, data processing, data analysis, and writing, with feedback on writing by co-authors.

\par I have strived to make this thesis adhere to the principle of open, reliable, reproducible, and transparent research.
As such, I have made all chapters of this thesis, and the code for the thesis itself, available as an Organization of repositories on Github (see https://github.com/phd-thesis-bpme).
Each of the data chapters below will have been contained in their own repository prior to being compiled into this thesis.
As such, I have ``forked" these existing repositories into the thesis organization previously linked, and have made necessary modifications to the code specific to this thesis.
These repositories will be archived upon final thesis submission, and this sentence will be replaced with a final DOI of all archived code for this thesis.

\par Chapters 1, 2, and parts of 6 are sections for a manuscript being prepared for peer-review at Ornithological Applications for a special feature entitled "Importance of detectability and recent advances in estimation".

\par Chapter 3 has been published in a peer-reviewed journal:
Edwards BPM, Smith AC, Docherty TGS, Gahbauer MA, Gillespie CR, Grinde AR, Harmer T, Iles DT, Matsuoka SM, Michel NL, Murray A, Niemi GJ, Pasher J, Pavlacky Jr DC, Robinson BG, Ryder TB, Sólymos P, Stralberg D, Zlonis EJ. (2023). Point Count Offsets for Estimating Population Sizes of North American Landbirds. Ibis 165(2):482-503. https://doi.org/10.1111/ibi.13169

\par Chapter 4 is under review at Ornithogical Applications for a special feature entitled "Importance of detectability and recent advances in estimation":
Edwards BPM, Gahbauer MA, Grinde A, Hope DD, Knight EC, Michel NL, Robinson BG, Sólymos P, Bennett JR, Smith AC. Predicting detection probabilities for rare and undersampled North American landbirds. Submitted to Ornithological Applications. [Under 1st revision, ORNITH-APP-24-159]

\par Chapter 5 is in being prepared for peer-review at Ecological Monographs.

\par All abstracts, published or otherwise, are contained in Chapter 1 under the ``Research Approach" subsection.



\clearpage

\pdfbookmark{Contents}{Contents}
\tableofcontents

\listoffigures

\listoftables


\label{endfrontmatter} % For automatic counting of pages
